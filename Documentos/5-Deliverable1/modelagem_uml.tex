\chapter{Modelagem UML}

\section{Requisitos funcionais}

\begin{enumerate}[topsep=0pt, partopsep=0pt, itemsep=0pt]
%  \item A estação base deve receber informações de posicionamento do robô, processá-las e armazená-las na memória. Representado pelo requisito funcional: \textbf{``Estação base mostra na interface gráfica a posição do robô - RF\arabic{enumi}"}
%  \item A estação base deve receber informações de obstáculos detectados, processá-las e armazená-las na memória. Representado pelo requisito funcional: \textbf{"Estação base mostra na interface gráfica os novos obstáculos detectados pelo robô - RF\arabic{enumi}"}
  \item A estação base deve mostrar na interface gráfica um mapa 2D (atualizado automaticamente) representando o robô e os obstáculos detectados por ele. Representado pelo requisito funcional: \textbf{``Estação base mostra mapa 2D do robô e dos obstáculos detectados -- RF\arabic{enumi}''}.
  \item O usuátio pode salvar o mapa 2D no disco rígido. Representado pelo requisito funcional: \textbf{``O usuário pode salvar o mapa -- RF\arabic{enumi}''}.
  \item O usuário pode carregar o mapa 2D do disco rígido. Representado pelo requisito funcional: \textbf{``O usuário pode carregar o mapa -- RF\arabic{enumi}''}.
  \item A estação base deve mostrar na interface gráfica a imagem captada pela \textit{webcam} do robô. Representado pelo requisito funcional: \textbf{``Estação base mostra a imagem captada pela webcam -- RF\arabic{enumi}''}.
  \item O usuário pode movimentar o robô, controlando a velocidade de suas rodas remotamente pelo teclado da estação base. Representado pelo requisito funcional: \textbf{``O usuário pode movimentar o robô -- RF\arabic{enumi}''}.
  \item O usuário pode parar o robô remotamente através de comandos do teclado da estação base. Representado pelo requisito funcional: \textbf{``O usuário pode parar o robô -- RF\arabic{enumi}''}.
  \item A estação base deve ser capaz de estabelecer conexão com o robô, informando o usuário caso a conexão ocorra com sucesso ou não. Representado pelo requisito funcional \textbf{``O usuário pode estabelecer a conexão entre o robô e a estação base -- RF\arabic{enumi}''}.
\end{enumerate}

\section{Requisitos não funcionais}

\begin{enumerate}[topsep=0pt, partopsep=0pt, itemsep=0pt]
  \item A imagem transmitida pela câmera do robô deve ser colorida. Representado pelo requisito não funcional: \textbf{``O robô deve enviar vídeo em imagem colorida para a estação base - RNF\arabic{enumi}''}.
  \item O robô deve transmitir as imagens de sua câmera em tempo real. Representado pelo requisito não funcional: \textbf{``O robô deve transmitir os dados de vídeo captados pela câmera em tempo real - RNF\arabic{enumi}''}.
%  \item O método de entrada do usuário deve se dar por meio de interface gráfica com o auxílio do mouse. Representado pelo requisito não funcional: \textbf{"O usuário pode interagir com o robô por meio do mouse - RNF0\arabic{enumi}"}.
%  \item O método de entrada do usuário deve se dar por meio de interface gráfica com o auxílio do teclado. Representado pelo requisito não funcional: \textbf{"O usuário pode interagir com o robô por meio do teclado - RNF0\arabic{enumi}"}.
\end{enumerate}


\section{Casos de uso identificados}

\begin{enumerate}[topsep=0pt, partopsep=0pt, itemsep=0pt]
  \item Visualização de mapa 2D na interface gráfica segundo os dados lidos dos sensores do robô. Representado pelo caso de uso: \textbf{``Mostrar mapa - UC\arabic{enumi}''}.
  \item Gravação do mapa em um arquivo no disco rígido. Representado pelo caso de uso: \textbf{``Salvar mapa - UC\arabic{enumi}''}. 
  \item Leitura do mapa de um arquivo do disco rígido. Representado pelo caso de uso: \textbf{``Carregar mapa - UC\arabic{enumi}''}.
  \item Leitura de informações dos sensores do robô. Representado pelo caso de uso: \textbf{``Leitura de sensores - UC\arabic{enumi}''}.
  \item Captura de imagens da \textit{webcam} do robô. Representado pelo caso de uso: \textbf{``Capturar imagens da câmera - UC\arabic{enumi}''}.
  \item Visualização de imagens da \textit{webcam} do robô. Representado pelo caso de uso: \textbf{``Visualizar imagens da câmera - UC\arabic{enumi}''}.
  \item Alteração pelo usuário da velocidade das rodas do robô. Representado pelo caso de uso: \textbf{``Movimentar o robô - UC\arabic{enumi}''}.
  \item Parada do robô solicitada pelo usuário. Representado pelo caso de uso: \textbf{``Parar o robô - UC\arabic{enumi}''}.
  \item Solicitação de estabelecimento de conexão com o robô. \textbf{``Estabelecer conexão - UC\arabic{enumi}''}.
%  \item Inclusão de novos obstáculos e suas posições lidos do robô. Representado pelo caso de uso: \textbf{``Mostrar posição dos novos obstáculos detectados na tela - UC\arabic{enumi}''}.
  \item Consulta da documentação do robô solicitada pelo usuário. Representado pelo caso de uso: \textbf{``Consultar documentação do robô - UC\arabic{enumi}''}.
\end{enumerate}

Na Figura \ref{fig:diagrama_casos_de_uso} está presente o diagrama de casos de uso, que foi produzido tomando-se como base as especificações apresentadas anteriormente. Vale ressaltar que as flechas pontilhadas (que denotam dependência entre casos de uso), apesar de deixarem o aspecto visual mais complexo, foram inseridas para reforçar a relação entre as funcionalidades da estação base e do robô.

\begin{figure}[H]
  \centering
  \includegraphics[width=\textwidth, keepaspectratio]{./figuras/diagrama_casos_de_uso.png}
  \caption{Diagrama de casos de uso.}
  \label{fig:diagrama_casos_de_uso}
\end{figure}

\section{Diagrama de classes}

A Figura \label{fig:diagrama_classes} mostra o diagrama de classes, tanto da estação base quanto do \textit{software} embarcado. A maior parte das classes pertence ao \textit{software} da estação base. As classes do \textit{software} do sistema embarcado que existem até o momento são as com prefixo \textit{Server} (no pacote \textit{comunicacao}). 

\begin{figure}[H]
  \centering
  \includegraphics[width=\textwidth]{./figuras/diagrama_classes.jpg}
  \caption{Diagrama de classes}
  \label{fig:diagrama_classes}
\end{figure}

\section{Descrição das classes}
%O software da estação base do robô foi dividido em cinco pacotes:  visual, controle, comunicação, controle.robo e interface gráfica. Estes serão descritos com suas respectivas classes na Tabela \ref{tab:pacote_visual}.
O software da estação base do robô foi dividido em cinco pacotes:  \textit{visual}, \textit{controle}, \textit{comunicao}, e \textit{gui}. A seguir há uma descrição de cada pacote e das suas respectivas classes.


\subsection{Pacote \textit{visual}}

Este pacote consiste de toda a parte visual da estação base e conta com as seguintes classes: Viewer2D, Drawable2D, EscalaDrawable, RoboDrawable, RoboTrilhaDrawable, ObstaculosDrawable, e Ponto. Na Tabela \ref{tab:pacote_visual} estão descritas as classes deste pacote.


\begin{table}[h]
  \centering
  \caption{Pacote \textit{visual}}
    \begin{tabular}{p{6cm}p{8cm}}
    \toprule
    \textbf{Classe} & \textbf{Descrição} \\
    \midrule
    Viewer2D & Responsável por exibir os objetos Drawable2D. Possui recursos de pan, zoom e rotate.   \\ \hline
    Drawable2D & Representa genericamente objetos 2D que podem ser desenhados em um Viewer2D. \\ \hline
    EscalaDrawable & Responsável por desenhar uma escala gráfica no mapa. \\ \hline
    RoboDrawable & Responsável por desenhar o robô no mapa. \\ \hline
    RoboTrilhaDrawable & Responsável por desenhar a trilha percorrida pelo robô no mapa. \\ \hline
    ObstaculosDrawable & Responsável por desenhar os pontos de cada obstáculo no mapa. \\ \hline
%    EscalaDrawableProp & Contém as propriedades visuais de desenho da escala. \\ \hline
%    RoboDrawableProp & Contém as propriedades visuais de desenho do robô \\ \hline
%    RoboTrilhaDrawableProp & Contém as propriedades visuais de desenho da trilha do robô. \\ \hline
%    ObstaculosDrawableProp & Contém as propriedades visuais de desenho dos obstáculos. \\ \hline
    Ponto & Representa um ponto de cordenadas cartesianas (x,y). \\ 
    \bottomrule
    \end{tabular}%
  \label{tab:pacote_visual}%
\end{table}%

\subsection{Pacote \textit{controle}}

Este pacote consiste de toda a parte da estação base que controla as informações essenciais do robô. Conta com as seguintes classes: Mapa, Obstaculos, Robo, ControleSensores, Posinfo, SensorIR e ControleCamera. Na Tabela \ref{tab:pacote_controle} estão descritas as classes deste pacote.

\begin{table}[h]
  \centering
  \caption{Pacote \textit{controle}}
    \begin{tabular}{p{6cm}p{8cm}}
    \toprule
    \textbf{Classe} & \textbf{Descrição} \\ 
    \midrule
    Mapa  & Responsável por representar o mapa. Armazena as informações essenciais do robô e dos obstáculos detectados. \\ \hline
    Obstaculos & Responsável por conter os obstáculos detectados pelo robô. \\ \hline
    Robo  & Responsável por representar o robô, este contêm largura, comprimento e centro de movimento (ponto central entre as duas rodas). \\ \hline
    ControleSensores & Responsável em atualizar a posição do robô e dos pontos que representão os obstáculos, de acordo com as leituras feitas pelos sensores. \\ \hline
    Posinfo & Responsável por conter as informações de uma posição do robô. \\ \hline
    SensorIR & Responsável por representar um sensor IR do robô. \\ 
%    ControleCamera & Responsável por controlar as imagens da câmera e o status do recebimento das imagens. \\ 
    \bottomrule
    \end{tabular}%
  \label{tab:pacote_controle}%
\end{table}%

\subsection{Pacote \textit{comunicacao}}

%Este pacote consiste em toda a parte de comunicação da estação base com o robô e conta com as seguintes classes: ClientCommandInterpreter, ClientConnection, ClientReceiver, ClientSender, ServerCommandInterpreter, ServerListener, ServerSender, ServerReceiver e Message. Na Tabela \ref{tab:pacote_comunicacao} estão descritas as classe deste pacote.
Este pacote consiste em toda a parte de comunicação da estação base com o robô e conta com as seguintes classes: ClientConnection, ClientReceiver, ClientSender, Server, ServerListener, ServerSender, ServerReceiver e Message. Na Tabela \ref{tab:pacote_comunicacao} estão descritas as classes deste pacote.

\begin{table}[h]
  \centering
  \caption{Pacote \textit{comunicacao}}
    \begin{tabular}{p{6cm}p{8cm}}
    \toprule
    \textbf{Classe} & \textbf{Descrição} \\ 
    \midrule
%    ClientCommandInterpreter & Responsável pela interpretação dos comandos do cliente. Os comandos recebidos são inseridos em uma fila, de modo a serem posteriormente executados pela thread. \\ \hline
    ClientConnection & Responsável por efetuar a gerência da conexão do cliente (estação base) com o servidor (robô). \\ \hline
    ClientReceiver & Responsável por receber mensagens de um host de uma conexão. \\ \hline
    ClientSender & Responsável por enviar mensagens ao host de uma conexão. \\ \hline
%    ServerCommandInterpreter & Responsável pela interpretação dos comandos do servidor. Os comandos recebidos são inseridos em uma fila, de modo a serem posteriormente executados pela thread. \\ \hline
    Server & Responsável gerenciar o servidor (robô). \\ \hline
    ServerListener & Responsável por escutar as novas conexões de clientes. \\ \hline
    ServerSender & Responsável por enviar mensagens ao host de uma conexão. \\ \hline
    ServerReceiver & Responsável por receber mensagens de um host de uma conexão. \\ \hline
    Message & Contém uma mensagem a ser enviada por um Sender. \\ 
    \bottomrule
    \end{tabular}%
  \label{tab:pacote_comunicacao}%
\end{table}%

\subsection{Pacote \textit{gui}}

Este pacote consiste em toda a interface gráfica do sistema e conta com as seguintes classes: JanelaConexao e JanelaPrincipal. Na Tabela \ref{tab:pacote_interface_grafica} estão descritas as classes deste pacote.

\begin{table}[h]
  \centering
  \caption{Pacote \textit{gui}}
    \begin{tabular}{p{6cm}p{8cm}}
    \toprule
    \textbf{Classe} & \textbf{Descrição} \\ 
    \midrule
    JanelaConexao & Janela com as informações e configurações da conexão com o Bellator. \\ \hline
    JanelaPrincipal & Janela principal da interface gráfica da estação base. \\ 
%    JanelaSensores & Responsável por desenhar a janela responsável pela configuração dos sensores. \\ 
    \bottomrule
    \end{tabular}%
  \label{tab:pacote_interface_grafica}%
\end{table}%

%\subsection{Pacote controle do robô}
%
%Este pacote consiste de toda a parte do sistema embarcado que gerencia os sensores e atuadores do robô. Conta com as seguintes classes: SensorsManager, WebcamManager e EnginesManager. Na Tabela \ref{tab:pacote_controle_robo} serão descritas as classes deste pacote.
%
%\begin{table}[h]
%  \centering
%  \caption{Pacote de controle de robô}
%    \begin{tabular}{p{6cm}p{8cm}}
%    \toprule
%    \textbf{Classe} & \textbf{Descrição} \\ 
%    \midrule
%    SensorsManager & Responsável por gerenciar a coleta de informações dos sensores do robô. \\ \hline
%    WebcamManager & Responsável por gerenciar a captura de imagens da webcam. \\ \hline
%    EnginesManager & Responsável por gerenciar a ação dos motores do robô. \\ 
%    \bottomrule
%    \end{tabular}%
%  \label{tab:pacote_controle_robo}%
%\end{table}%
%
%

