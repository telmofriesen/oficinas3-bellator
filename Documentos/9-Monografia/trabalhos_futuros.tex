\chapter{Trabalhos futuros}
Há inúmeros modos de se prosseguir com o projeto do robô Bellator, pois muitas melhorias podem ser efetuadas a fim de se atingir resultados melhores em termos de mapeamentos de ambientes.

No que diz respeito ao sensoriamento, é importante melhorar a confiabilidades dos dados dos encoders. Uma das formas de se fazer isso é utilizando um \textit{timing belt}, que é uma roda dentada acoplada a um sensor. Dessa forma, problemas de escorregamento da correia em relação à polias do robô seriam eliminados.

Quanto ao uso do acelerômetro, a técnica do filtro de Kalman poderia ser explorada se forma a se realizar a fusão de dados dos três sensores do robô: encoder, acelerômetro e giroscópio. Se desenvolvida corretamente, em tese, a utilização desta técnica permitiria a compensação automática de erros dos sensores o que permitiria uma estimativa muito melhor do posicionamento do robô. As dificuldades de se utilizar esta técnica são: correta determinação da margem de erro de cada sensor, criação de um modelo matemático da movimentação do robô e complexidade do entendimento e implementação do filtro.

Em relação à câmera, há dois aspectos a serem melhorados: a redução do atraso do sinal de vídeo (que é da ordem de segundos) e a interação do usuário com a câmera. O primeiro problema poderia ser resolvido com a substituição da \textit{webcam} atual por outra que possua menos latência, ou pelo desenvolvimentos de software que faça a leitura dos dados brutos da câmera e faça o envio instantâneo desses dados à estação base, procurando reduzir ou eliminar a utilização de \textit{buffers} de vídeo. Em termos de interatividade, pode-se construir um dispositivo de rotação que permita ao usuário girar a câmera 360 graus em torno do eixo vertical, possibilitando maior flexibilidade e autonomia durante a utilização do robô.