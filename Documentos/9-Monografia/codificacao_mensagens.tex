\subsection{Codificação das mensagens}
\label{sec:codificacao_mensagens}

\begin{itemize}
  \item Mensagens do TS-7260 para o LPC2103 (via porta serial)
    	
	\begin{itemize}
		
	  \item \textbf{SYNC (0xA0)}\\
	  Quando o microcontrolador LPC2103 recebe esta mensagem, responde com as leituras mais recentes dos encoders, de cada sensor de distância, do acelerômetro e do giroscópio (enviando a mensagem SENSORS, explicada abaixo).
	  
	  \item \textbf{ENGINES (0xB0)}\\
	  \textit{(byte) vel\_roda\_esquerda}\\
	  \textit{(byte) vel\_roda\_direita}\\
	  \textit{(byte) CHECKSUM\_H}\\
	  \textit{(byte) CHECKSUM\_L}\\
	  Ao receber este comando, o microcontrolador utiliza os valores para definir o nível de PWM para as rodas do robô. Os valores de velocidade são representados por um byte cada, nos quais o bit mais significativo indica o sentido de rotação da roda (1 para frente e 0 para trás) e os restantes a intensidade do PWM.
	  
	  Os bytes de checksum são utilizados para verificar se não há dados corrompidos. Os bytes da mensagem são somados (módulo 65536) e o resultado é atribuído aos bytes (high e low) de checksum.
	  \end{itemize}
	  
	  \item Mensagens do LPC2103 para a TS-7260 (via porta serial)
	  
	  \begin{itemize}

	  \item \textbf{ENGINES\_ACK (0xB1)}\\
	  \textit{(byte) vel\_roda\_esquerda}\\
	  \textit{(byte) vel\_roda\_direita}\\
	  \textit{(byte) CHECKSUM\_H}\\
	  \textit{(byte) CHECKSUM\_L}\\
	  Esta mensagem deve ser enviada toda vez que um comando de mudança de velocidade (ENGINES) for recebido na placa de baixo nível. A mensagem é usada no Linux embarcado para verificar se o comando foi corretamente recebido. Caso uma confirmação não seja recebida em certo intervalo de tempo, outro comando ENGINES é enviado para a placa de baixo nível.
	 
	  O checksum é tem função idêntica ao que já foi explicitado na mensagem ENGINES.
	  
	  \item \textbf{SENSORS (0xC0)}\\
	  \textit{(byte) encoder\_esq\_H}, \textit{(byte) encoder\_esq\_L},\\
	  \textit{(byte) encoder\_dir\_H}, \textit{(byte) encoder\_dir\_L},\\
	  \textit{(byte) IR1}, \textit{(byte) IR2}, \textit{(byte) IR3}, \textit{(byte) IR4}, \textit{(byte) IR5},\\
	  \textit{(byte) AX\_H}, \textit{(byte) AX\_L},\\
	  \textit{(byte) AY\_H}, \textit{(byte) AY\_L},\\
	  \textit{(byte) AZ\_H}, \textit{(byte) AZ\_L},\\
	  \textit{(byte) GX\_H}, \textit{(byte) GX\_L},\\
	  \textit{(byte) GY\_H}, \textit{(byte) GY\_L},\\
	  \textit{(byte) GZ\_H}, \textit{(byte) GZ\_L},\\
	  \textit{(byte) TIMESTAMP\_H}, \textit{(byte) TIMESTAMP\_L}\\
	  \textit{(byte) CHECKSUM\_H}\\
	  \textit{(byte) CHECKSUM\_L}\\
	  Representa a leitura de todos os sensores (encoders, infra-vermelhos, acelerômetro e giroscópio). 
	  
	  Os 4 primeiros bytes são os valores das leituras dos encoders esquerdo e direíto (cada um com um byte alto e um baixo). Os valores das leituras dos encoders representam a diferença entre a contagem atual a contagem anterior.
	  
	  Nos próximos 5 bytes, as leituras do sensores ópticos são enviadas em sequência. As distâncias que os sensores ópticos são capazes de mensurar são dividos em valores discretos de 0 a 255 \cite{bellator_2012}. 
	  
	  Após isso, os 12 bytes que se seguem representam as leituras do acelerômetro e do giroscópio. Os bytes que começam com `A' representam a leitura de cada um dos eixos do acelerômetro. Aqueles que começam com `G' representam a leitura de cada um dos eixos do giroscópio.
	  
	  O timestamp (valor alto e baixo) é um contador de 16 bits que é incrementado entre cada amostra e zera automaticamente depois que chega ao valor máximo (65535), usado para determinar o instante em que foi feita a leitura dos dados. Como a amostragem dos sensores na placa de baixo nível será efetuada em intervalos fixos, a informação do contador do timestamp pode ser utilizada para obter informações de tempo de cada amostra.

	  O checksum é tem função idêntica ao que já foi explicitado na mensagem ENGINES.
	  
	  
	\end{itemize}

  \item Mensagens bidirecionais entre estação base e TS-7260 (via Wi-Fi):

    \begin{itemize}
      \item \textbf{ECHO\_REQUEST (0x01)}\\
      \textit{(byte) END\_CMD}\\
	Requisição de ping.
      \item \textbf{ECHO\_REPLY (0x02)}\\
      \textit{(byte) END\_CMD}\\
	Resposta de ping.
      \item \textbf{DISCONNECT (0x0F)} \\
	Solicitação de desconexão.
    \end{itemize}

  \item Mensagens da estação base para a TS-7260 (via Wi-Fi):

    \begin{itemize}
      \item \textbf{HANDSHAKE\_REQUEST (0x10)}\\
	Solicitação de handshake.

      \item \textbf{HANDSHAKE\_CONFIRMATION (0x12)}\\
	Confirmação de handshake.

      \item \textbf{SENSORS\_START (0x20)}\\
	Solicitação de início da amostragem dos sensores.

      \item \textbf{SENSORS\_STOP (0x21)}\\
	Solicitação de parada da amostragem dos sensores.

      \item \textbf{SENSORS\_RATE (0x22)} \\
	\textit{(float) Nova taxa de amostragem (comandos SYNC por segundo)}\\
	Solicitação de mudança da taxa de envio de comandos SYNC da TS para a placa de baixo nível.

%       \item \textbf{SENSORS\_STATUS\_REQUEST}\\
%       \textit{(byte) END\_CMD}\\
% 	Requisição de status da amostragem dos sensores. Usado na interface gráfica para atualizar as informações sobre os sensores.

      \item \textbf{WEBCAM\_START (0x30)}\\
	Solicitação de início da amostragem da webcam.

      \item \textbf{WEBCAM\_STOP (0x31)}\\
	Solicitação de parada da amostragem da webcam.

      \item \textbf{WEBCAM\_RATE (0x32)} \\
	\textit{(float) Nova taxa de quadros}\\
	Solicitação de mudança da taxa de quadros da webcam.

      \item \textbf{WEBCAM\_RESOLUTION (0x33)} \\
	\textit{(int) Largura em pixels }\\
	\textit{(int) Altura em pixels}\\
	Solicitação de mudança da resolução da webcam.

%       \item \textbf{WEBCAM\_STATUS\_REQUEST}\\
%       \textit{(byte) END\_CMD}\\
% 	Solicitação de informações sobre status da webcam. Usado na interface gráfica para atualizar as informações sobre a webcam.

      \item \textbf{ENGINES (0xB0)} \\
	 \textit{(float) vel\_roda\_esquerda}\\
	 \textit{(float) vel\_roda\_direita}\\
	Solicitação de mudança da velocidade dos motores. Para cada roda há um valor de -1 até 1, sendo que -1 é a máxima velocidade para trás, 1 a máxima velocidade para frente e 0 é parada da roda.

%       \item \textbf{ENGINES\_STATUS\_REQUEST}\\
%       \textit{(byte) END\_CMD}\\
% 	Solicitação de status dos motores. Usado na interface gráfica para confirmar o recebimento de comandos de movimentação efetuados pelo usuário.

    \end{itemize}

  \item Mensagens da TS-7260 para a estação base (via Wi-Fi):

    \begin{itemize}
      \item \textbf{HANDSHAKE\_REPLY (0x11)}\\
	Resposta de handshake.
	
	 \item \textbf{SENSORS (0xC0)}\\
	  \textit{(byte) encoder1\_H}, \textit{(byte) encoder1\_L},\\
	  \textit{(byte) encoder2\_H}, \textit{(byte) encoder2\_L},\\
	  \textit{(byte) IR1}, \textit{(byte) IR2}, \textit{(byte) IR3}, \textit{(byte) IR4}, \textit{(byte) IR5},\\
	  \textit{(byte) AX\_H}, \textit{(byte) AX\_L},\\
	  \textit{(byte) AY\_H}, \textit{(byte) AY\_L},\\
	  \textit{(byte) AZ\_H}, \textit{(byte) AZ\_L},\\
	  \textit{(byte) GX\_H}, \textit{(byte) GX\_L},\\
	  \textit{(byte) GY\_H}, \textit{(byte) GY\_L},\\
	  \textit{(byte) GZ\_H}, \textit{(byte) GZ\_L},\\
	  \textit{(byte) TIMESTAMP\_H}, \textit{(byte) TIMESTAMP\_L}\\
	  
	  Possui a mesma funcionalidade e parâmetros que a mensagem SENSORS enviada da LPC2103 para a TS-7260, com exceção do timestamp, que é trocado por um timestamp UNIX em milissegundos (que representa o horário absoluto em que a amostra foi obtida na placa de baixo nível). Essa informação de tempo é utilizada pela estação base para efetuar os cálculos de posicionamento do robô.

      \item \textbf{SENSORS\_STATUS (0xC1)} \\
	\textit{(boolean) Status da amostragem [on - off] }\\
	\textit{(float) Taxa de amostragem}\\
	Informações de status dos sensores. Usado na interface gráfica para confirmar o recebimento de comandos de mudança de taxa de amostragem e início/parada da amostragem.

      \item \textbf{WEBCAM\_STATUS (0x34)} \\
% 	\textit{(boolean) Nova taxa de amostragem }\\
	\textit{(float) Taxa de quadros }\\
	\textit{(int) Largura em pixels }\\
	\textit{(int) Altura em pixels }\\
	\textit{(boolean) Status da stream [on - off] }\\
	\textit{(int) Porta da stream}\\
	Informações de status da webcam. Usado na interface gráfica para confirmar o recebimento de comandos relativos à webcam, e para que a estação base tenha conhecimento do status da stream da webcam.
	
      \item \textbf{ENGINES\_STATUS (0xB1)} \\
	\textit{(byte) vel\_roda\_esquerda}\\
	\textit{(byte) vel\_roda\_direita}\\
	Informações sobre as velocidades programadas dos motores. Usado na interface gráfica para confirmar o recebimento de comandos de movimentação efetuados pelo usuário.

	

    \end{itemize}
\end{itemize}
