\documentclass{beamer}
\usecolortheme{dove}
% \usepackage[T1]{fontenc}
% \usepackage{ucs}
\usepackage[utf8x]{inputenc}





\begin{document}

  %-----------------------------
  \begin{frame}
    \frametitle{Codificação das mensagens\\ (Linux Embarcado $\rightarrow$ Placa de baixo nível)}
	\begin{itemize}
	  \item \textbf{SYNC}\\
	  \textit{(byte) END\_CMD}\\
	  Quando o microcontrolador LPC2103 recebe esta mensagem, responde com as leituras mais recentes de cada sensor de distância, dos encoders, do acelerômetro e do giroscópio.
	  \item \textbf{LEFT\_WHEEL}\\
	  \textit{(byte) valor}\\
	  \textit{(byte) END\_CMD}\\
	  Ao receber este comando, o microcontrolador utiliza o valor para definir o nível de PWM para a roda esquerda do robô. Valor é representado por apenas um byte, onde o bit mais significativo indica o sentido de rotação da roda e os restantes a intensidade do PWM;
	  
	  \end{itemize}
	\end{frame}
	\begin{frame}
	\frametitle{Codificação das mensagens\\ (Linux Embarcado $\rightarrow$ Placa de baixo nível)}
	  \begin{itemize}	  
	  \item \textbf{RIGHT\_WHEEL}\\
	  \textit{(byte) valor}\\
	  \textit{(byte) END\_CMD}\\
	  Funcionamento idêntico ao comando LEFT\_WHEEL, e mas para a roda direita.
	  
	   \end{itemize}
	\end{frame}
	\begin{frame}
	\frametitle{Codificação das mensagens\\ (Linux Embarcado $\leftarrow$ Placa de baixo nível)}
	  \begin{itemize}
	  
	  \item \textbf{OPTICAL\_SENSOR}\\
	   \textit{(byte) Número do sensor}\\
	   \textit{(byte) Distância medida}\\
	   \textit{(byte) END\_CMD}\\
	   Representa a leitura de cada sensor, onde valor é um byte, cuja faixa de variação e [0, 255].
	   
	  \item \textbf{ENCODER}\\
	  \textit{(byte) Número do encoder}\\
	  \textit{(byte) valor\_high}\\
	  \textit{(byte) valor\_low}\\
	  \textit{(byte) END\_CMD}\\
	  Representa a leitura de cada encoder, valor\_high e valor\_low juntos formam um inteiro de 16 bits que contém o valor da contagem do encoder.
	  
	  \end{itemize}
	\end{frame}
	\begin{frame}
	\frametitle{Codificação das mensagens\\ (Linux Embarcado $\leftarrow$ Placa de baixo nível)}
	  \begin{itemize}
	  \item \textbf{ACEL\_GYRO}\\
	  \textit{(byte) TIMESTAMP}\\
	  \textit{(byte) AX\_H}, \textit{(byte) AX\_L}\\
	  \textit{(byte) AY\_H}, \textit{(byte) AY\_L}\\
	  \textit{(byte) AZ\_H}, \textit{(byte) AZ\_L}\\
	  \textit{(byte) GX\_H}, \textit{(byte) GX\_L}\\
	  \textit{(byte) GY\_H}, \textit{(byte) GY\_L}\\
	  \textit{(byte) GZ\_H}, \textit{(byte) GZ\_L}\\
	  \textit{(byte) END\_CMD}\\
	  Representa a leitura do acelerômetro e gisroscópio. Os bytes que começam com A representam a leitura de um dos eixos do acelerômetro. Aqueles que começam com G representam a leitura de um dos eixos do giroscópio.
	\end{itemize}
  \end{frame}
  
  
  \begin{frame}
    \frametitle{Codificação das mensagens\\ (Estação Base $\leftrightarrow$ Linux Embarcado)}
    \begin{itemize}
      \item \textbf{ECHO\_REQUEST}\\
	Requisição de ping.
      \item \textbf{ECHO\_REPLY}\\
	Resposta de ping.
      \item \textbf{DISCONNECT} \\
	Solicitação de desconexão.
    \end{itemize}
  \end{frame}
  
  %--------------------
  
    \begin{frame}
    \frametitle{Codificação das mensagens\\ (Estação Base $\rightarrow$ Linux Embarcado)}
		\begin{itemize}
		\item \textbf{HANDSHAKE\_REQUEST}\\
		Solicitação de handshake.

		\item \textbf{HANDSHAKE\_CONFIRMATION}\\
		Confirmação de handshake.

		\item \textbf{SENSORS\_START}\\
		Solicitação de início da amostragem dos sensores.

		\item \textbf{SENSORS\_STOP}\\
		Solicitação de parada da amostragem dos sensores.

		\item \textbf{SENSORS\_RATE} \\
		\textit{(float) Nova taxa de amostragem (amostras/s)}\\
		Solicitação de mudança da taxa de amostragem dos sensores.
		
		\end{itemize}
	\end{frame}
	\begin{frame}
	\frametitle{Codificação das mensagens\\ (Estação Base $\rightarrow$ Linux Embarcado)}
		\begin{itemize}

		\item \textbf{SENSORS\_STATUS\_REQUEST}\\
		Solicitação de informações sobre status dos sensores.

		\item \textbf{WEBCAM\_START}\\
		Solicitação de início da amostragem da webcam.

		\item \textbf{WEBCAM\_STOP}\\
		Solicitação de parada da amostragem da webcam.

		\item \textbf{WEBCAM\_RATE} \\
		\textit{(float) Nova taxa de quadros por segundo}\\
		Solicitação de mudança da taxa de quadros por segundo (fps) da webcam.
		
		\end{itemize}
	\end{frame}
	\begin{frame}
	\frametitle{Codificação das mensagens\\ (Estação Base $\rightarrow$ Linux Embarcado)}
		\begin{itemize}

		\item \textbf{WEBCAM\_RESOLUTION} \\
		\textit{(int) Largura em pixels }\\
		\textit{(int) Altura em pixels}\\
		Solicitação de mudança da resolução da webcam.

		\item \textbf{WEBCAM\_STATUS\_REQUEST}\\
		Solicitação de informações sobre status da webcam.

		\item \textbf{ENGINES\_SPEED} \\
		\textit{(byte) Nova velocidade da roda esquerda (Valor de 0 a 255) }\\
		\textit{(byte) Nova velocidade da roda direita (Valor de 0 a 255)}\\
		Solicitação de mudança da velocidade dos motores.

		\item \textbf{ENGINES\_STATUS\_REQUEST}\\
		Solicitação de informações sobre status dos motores.
		\end{itemize}
  \end{frame}
  
  %-------------------------------------
  \begin{frame}
	\frametitle{Codificação das mensagens\\ (Estação Base $\leftarrow$ Linux Embarcado)}
	\begin{itemize}
		\item \textbf{HANDSHAKE\_REPLY}\\
		Resposta de handshake.

		\item \textbf{SENSORS\_STATUS} \\
		\textit{(boolean) Status da amostragem [on - off] }\\
		\textit{(float) Taxa de amostragem}\\
		Informações de status dos sensores.

		\item \textbf{WEBCAM\_STATUS} \\
% 		\textit{(boolean) Nova taxa de amostragem }\\
		\textit{(float) Taxa de quadros }\\
		\textit{(int) Largura em pixels }\\
		\textit{(int) Altura em pixels }\\
		\textit{(boolean) Status da stream [on - off] }\\
		\textit{(int) Porta da stream}\\
		Informações de status da webcam.
		
		\end{itemize}
	\end{frame}
	\begin{frame}
	\frametitle{Codificação das mensagens\\ (Estação Base $\leftarrow$ Linux Embarcado)}
		\begin{itemize}
			
		\item \textbf{ENGINES\_STATUS} \\
		\textit{(byte) Velocidade programada da roda esquerda (Valor de 0 a 255) }\\
		\textit{(byte) Velocidade programada da roda direita (Valor de 0 a 255)}\\
		Informações de status dos motores.

		\item \textbf{ENCODERS} \\
		\textit{(int) Leitura roda esquerda }\\
		\textit{(int) Leitura roda direita }\\
		\textit{(long) Timestamp UNIX em milissegundos}\\
		Envio de leituras dos encoders.
		
		\end{itemize}
	\end{frame}
	\begin{frame}
	\frametitle{Codificação das mensagens\\ (Estação Base $\leftarrow$ Linux Embarcado)}
		\begin{itemize}

		\item \textbf{ACEL\_GYRO} \\
		\textit{(int) Aceleração em X }\\
		\textit{(int) Aceleração em Y }\\
		\textit{(int) Aceleração em Z }\\
		\textit{(int) Aceleração angular em X }\\
		\textit{(int) Aceleração angular em Y }\\
		\textit{(int) Aceleração angular em Z }\\
		\textit{(long) Timestamp UNIX em milissegundos}\\
		Envio de leituras do acelerômetro e giroscópio.

		\item \textbf{OPTICAL\_SENSORS} \\
% 		\textit{(byte) Número do sensor infra-vermelho} \\
		\textit{(byte[]) Distâncias detectada pelos sensores} \\
		\textit{(long) Timestamp UNIX em milissegundos}\\
		Envio de leituras dos sensores ópticos.

		\end{itemize}
	\end{frame}
  
\end{document}
