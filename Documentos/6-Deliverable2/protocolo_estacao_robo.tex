\begin{itemize}
  \item Mensagens do TS-7260 para o LPC2103
		\begin{itemize}
	  \item \textbf{SYNC\_END\_CMD}: quando o microcontrolador LPC2103 recebe esta mensagem, responde com as leituras mais recentes de cada sensor de distância, em seguida as leituras a dos encoders;
	  \item \textbf{LEFT\_WHEEL}: ao receber este comando, o microcontrolador utiliza o valor para definir o nível de PWM para a roda esquerda do robô valor e representado por apenas um byte, onde o bit mais significativo indica o sentido de rotação da roda e os restantes a intensidade do PWM;
	  \item \textbf{RIGHT\_WHEEL}: funcionamento idêntico ao comando LEFT\_WHEEL, e mas para a roda direita.
	  \item \textbf{OPTICAL\_SENSOR}:: representa a leitura de cada sensor, onde valor e um byte, cuja faixa de variação e [0, 255].
	  \item \textbf{ENCODER}: representa a leitura de cada encoder, valor high e valor low juntos formam um inteiro de 16 bits que contém o valor da contagem do encoder.
	  \item \textbf{IMU\_DATA}: representa a leitura de cada sensor. O processo possui uma \textit{timestamp}, para controlar a perda de dados. Os dados são lidos byte a byte. Os bytes que terminam com \_H representam a leitura dos bits mais significativos. Aqueles que terminam em \_L representam a leitura dos bits menos significativos. Aqueles que começam com A representam a leitura de um dos eixos do acelerômetro. Aqueles que começam com G representam a leitura de um dos eixos do giroscópio. A sequência ocorre da seguinte maneira: IMU\_DATA; TIMESTAMP; AX\_H; AX\_L; AY\_H; AY\_L; AZ\_H; AZ\_L; GX\_H; GX\_L; GY\_H; GY\_L; GZ\_H; GZ\_L; END\_CMD;
	\end{itemize}

  \item Mensagens bidirecionais:

    \begin{itemize}
      \item \textbf{ECHO\_REQUEST}\\
	Requisição de ping.
      \item \textbf{ECHO\_REPLY}\\
	Resposta de ping.
      \item \textbf{DISCONNECT} \\
	Solicitação de desconexão.
    \end{itemize}

  \item Mensagens da estação base para o robô:

    \begin{itemize}
      \item \textbf{HANDSHAKE\_REQUEST}\\
	Solicitação de handshake.

      \item \textbf{HANDSHAKE\_CONFIRMATION}\\
	Confirmação de handshake.

      \item \textbf{SENSORS\_START}\\
	Solicitação de início da amostragem dos sensores.

      \item \textbf{SENSORS\_STOP}\\
	Solicitação de parada da amostragem dos sensores.

      \item \textbf{SENSORS\_RATE} \\
	\textit{(float) Nova taxa de amostragem}\\
	Solicitação de mudança da taxa de amostragem dos sensores.

      \item \textbf{SENSORS\_STATUS\_REQUEST}\\
	Resposta de status da amostragem dos sensores.

      \item \textbf{WEBCAM\_START}\\
	Solicitação de início da amostragem da webcam.

      \item \textbf{WEBCAM\_STOP}\\
	Solicitação de parada da amostragem da webcam.

      \item \textbf{WEBCAM\_RATE} \\
	\textit{(float) Nova taxa de quadros}\\
	Solicitação de mudança da taxa de quadros da webcam.

      \item \textbf{WEBCAM\_RESOLUTION} \\
	\textit{(int) Largura em pixels }\\
	\textit{(int) Altura em pixels}\\
	Solicitação de mudança da resolução da webcam.

      \item \textbf{WEBCAM\_STATUS\_REQUEST}\\
	Solicitação de status da amostragem da webcam.

      \item \textbf{ENGINES\_SPEED} \\
	\textit{(int) Nova velocidade da roda esquerda (Valor de 0 a 255) }\\
	\textit{(int) Nova velocidade da roda direita (Valor de 0 a 255)}\\
	Solicitação de mudança da velocidade dos motores.

      \item \textbf{ENGINES\_STATUS\_REQUEST}\\
	Solicitação de status dos motores.

    \end{itemize}

  \item Mensagens do robô para a estação base:

    \begin{itemize}
      \item \textbf{HANDSHAKE\_REPLY}\\
	Resposta de handshake.

      \item \textbf{SENSORS\_STATUS\_REPLY} \\
	\textit{(boolean) Status da amostragem [on - off] }\\
	\textit{(float) Taxa de amostragem}\\
	Resposta de status da amostragem dos sensores.

      \item \textbf{WEBCAM\_STATUS\_REPLY} \\
	\textit{(boolean) Nova taxa de amostragem }\\
	\textit{(float) Taxa de quadros }\\
	\textit{(int) Largura em pixels }\\
	\textit{(int) Altura em pixels }\\
	\textit{(boolean) Status da stream [on - off] }\\
	\textit{(int) Porta da stream}\\
	Resposta de status da amostragem da webcam.

      \item \textbf{ENGINES\_STATUS\_REPLY} \\
	\textit{(int) Velocidade programada da roda esquerda (Valor de 0 a 255) }\\
	\textit{(int) Velocidade programada da roda direita (Valor de 0 a 255)}\\
	Resposta de status dos motores.

      \item \textbf{ENCODERS} \\
	\textit{(int) Leitura roda esquerda }\\
	\textit{(int) Leitura roda direita }\\
	\textit{(long) Timestamp em milissegundos}\\
	Envio de leituras dos encoders.

      \item \textbf{ACEL\_GYRO} \\
	\textit{(float) Aceleração em X }\\
	\textit{(float) Aceleração em Y }\\
	\textit{(float) Aceleração em Z }\\
	\textit{(float) Aceleração angular em X }\\
	\textit{(float) Aceleração angular em Y }\\
	\textit{(float) Aceleração angular em Z }\\
	\textit{(long) Timestamp em milissegundos}\\
	Envio de leituras do acelerômetro e giroscópio.

      \item \textbf{OPTICAL\_SENSORS} \\
	\textit{(float[]) Distâncias detectadas pelos sensores ópticos} \\
	\textit{(long) Timestamp em milissegundos}\\
	Envio de leituras dos sensores ópticos.

    \end{itemize}
\end{itemize}
