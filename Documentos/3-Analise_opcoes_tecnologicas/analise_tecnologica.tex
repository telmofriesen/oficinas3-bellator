\chapter{Análise tecnológica}
Nesta seção está explicitada, primeiramente, uma visão geral do projeto. Em seguida, há uma discussão detalhada a respeito dos requisitos de cada parte fundamental (estação base, sistema de comunicação, sistema embarcado). Por fim há uma enumeração das alternativas tecnológicas pesquisadas e das escolhidas para o preenchimento dos requisitos.

\section{Visão geral do projeto}

O projeto, como foi idealizado, consiste em um robô controlado manualmente capaz de efetuar mapeamento 2D de ambientes. A estação base é um computador controlado e monitorado por um usuário humano. O utilizador será capaz de enviar comandos de movimentação ao robô (via teclado) e receber \textit{feedback} do seu posicionamento e dos obstáculos detectados por ele. Além disso, as imagens em tempo real de uma câmera posicionada no robô -- aspecto explicado mais à frente -- poderão ser visualizadas pelo utilizador.

O sistema de comunicação deverá ter alcance máximo de 20 metros. Visto que toda a comunicação entre a estação base e o sistema embarcado será feita por um único canal, a velocidade de transmissão de dados deve ser suficiente para o envio de comandos de movimentação ao robô, recebimento de dados de leituras de sensores e recebimento de imagens da câmera em tempo real.

O sistema embarcado é, primariamente, o robô. Ele deve ser capaz de se mover para frente e para trás e girar para a esquerda e direita -- em velocidades não muito altas, o que é suficiente, visto que velocidades elevadas dificultam o controle de movimentação pelo usuário. Uma visualização em tempo real do ambiente pelo usuário, tendo o objetivo de facilitar o controle de movimentação manual, poderá ser feita através de imagens geradas por uma câmera fixa instalada no robô. 

O robô deve ser capaz de obter dados para cálculos (na estação base) da sua velocidade e deslocamento. Erros de medição em decorrência de escorregamento ou trepidação de rodas devem ser atenuados, visando dessa forma a futura utilização do robô em condições não ideais de terreno. Obstáculos próximos -- em uma distância mínima de 30 cm e máxima de 150 cm -- devem ser detectados de modo a possibilitar a confecção do mapa 2D em tempo real na estação base.

\section{Requisitos}


\subsection{Estação base}
%
%TODO verificar objetivos do projeto
Esta seção descreve os requisitos da estação base, que foram elaborados de forma a satisfazer os objetivos do projeto.

\begin{itemize} %-------------------

  \item O \textit{software} será executado em um computador pessoal.
    \begin{itemize}
      \item O programa deverá ser executado razoavelmente em computador com recursos de \textit{hardware} comparável aos padrões atuais (pelo menos 2GB de memória RAM DDR2 ou melhor e processador \textit{dual-core}).
      \item O \textit{software} deverá ser multiplataforma, ou seja, executar em diferentes sistemas operacionais (no mínimo Linux e Windows).
      \item Preferencialmente bibliotecas e ferramentas livres (e gratuitas) deverão ser utilizadas no desenvolvimento do \textit{software}.
    \end{itemize}

  \item O \textit{software} deve possuir uma interface gráfica.
    \begin{itemize}
      \item Um utilizador, através da interface gráfica, será capaz de controlar o robô enviando comandos de movimentação especificados pelo teclado. 
      \item O usuário receberá a imagem em tempo real (preferencialmente com atrasos não muito consideráveis) de uma câmera fixa instalada no robô. 
      \item Os dados instantâneos de velocidade e posição do robô serão mostrados ao usuário na interface gráfica.
      \item Um mapa 2D do caminho percorrido e dos obstáculos detectados pelo robô será gerado, na interface gráfica, à medida em que houver movimentação do mesmo. O caminho percorrido pelo robô será representado visualmente por pontos, gradualmente posicionados no mapa. Os obstáculos serão representados por marcações nas localidades onde houve detecção de objetos por sensores. Todos os pontos representados no mapa serão gerados a partir de amostras obtidas em intervalos de tempo discretos.
      \item O mapa 2D gerado na interface poderá ser salvo em um arquivo, podendo ser posteriormente carregado.
    \end{itemize}

\end{itemize} %----------------------



\subsection{Sistema de comunicação}
Esta seção descreve os requisitos do sistema de comunicação entre a estação base e o sistema embarcado.

\begin{itemize} %----------------------

  \item Distância entre robô e estação base.
    \begin{itemize}
      \item O sistema de comunicação deve possuir, ao menos, alcance máximo de 20 metros sem fios -- de modo que ambientes de tamanho razoável possam ser mapeados.
    \end{itemize}

  \item Velocidade e direção do fluxo de transmissão de dados.
    \begin{itemize}
      \item Toda a comunicação entre a estação base e o sistema embarcado será feita por um único canal e, portanto:
      \item A velocidade de transmissão do canal de comunicação deve ser suficiente para o envio de comandos de movimentação ao robô, recebimento de dados de leituras de sensores e recebimento de imagens da câmera em tempo real; 
      \item O fluxo de dados deve ser bidirecional (\textit{full-duplex}).
    \end{itemize}

  \item Complexidade de implementação.
    \begin{itemize}
      \item A tecnologia utilizada para a comunicação deve permitir fácil utilização dos protocolos de transporte TCP e UDP, simplificando o desenvolvimento do protocolo da camada de aplicação.
    \end{itemize}

\end{itemize} %----------------------



\subsection{Sistema embarcado}
Esta seção descreve os requisitos do sistema embarcado (robô).

\begin{itemize} %----------------------

  \item Movimentação do robô.
    \begin{itemize}
      \item O robô deve ser capaz de mover-se para frente, para trás e girar para a esquerda e direita.
%      \item A velocidade de movimentação pode ser relativamente pequena. %%TODO verificar velocidade do bellator nas monografias anteriores
    \end{itemize}

  \item Controle de posicionamento e velocidade.
    \begin{itemize}
      \item O robô deve ser capaz de obter dados que permitam calcular sua velocidade (linear e angular) e posição atual (deslocamento e rotação em relação à posição inicial). Deve ser capaz de enviar os dados à estação base.
    \end{itemize}

  \item Detecção de obstáculos.
    \begin{itemize}
      \item O robô deverá ser capaz de detectar obstáculos próximos -- com distância de no mínimo 30 cm e no máximo 150 cm -- localizados ao seu redor, determinando a distância de cada objeto detectado.
    \end{itemize}

\end{itemize} %----------------------


\chapter{Análise de opções tecnológicas}
%Alternativas de hardware
\input{opcoes_tecnologicas-hw} 
%Alternativas de software
\input{opcoes_tecnologicas-sw}