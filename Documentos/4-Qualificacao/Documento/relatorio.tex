% \documentclass{report}
% \usepackage[utf8]{inputenc}
% \usepackage[brazil]{babel}

\documentclass[oneside,a4paper,12pt]{normas-utf-tex}

\usepackage{breakurl}
\usepackage[alf,abnt-emphasize=bf,bibjustif,recuo=0cm, abnt-etal-cite=2]{abntcite}
\usepackage[brazil]{babel}
\usepackage[utf8]{inputenc}
\usepackage{amsmath}
\usepackage{graphicx,subfig}
\usepackage{times}
\usepackage[plain]{fancyref}
\usepackage{float}
\usepackage{pdfpages}
\usepackage{enumitem}

%%% Complemento para tabelas
\usepackage{booktabs, multirow}
\setlength{\heavyrulewidth}{0.1em}
\renewcommand{\toprule}{\midrule[\heavyrulewidth]}
\renewcommand{\arraystretch}{1.2}
%%%

\instituicao{Universidade Tecnológica Federal do Paraná}
\departamento{Departamento Acadêmico de Eletrônica}
\departamentodois{Departamento Acadêmico de Informática}
\programa{Curso de Engenharia de Computação}
\unidade{Oficina de Integração 3}

\titulo{\MakeUppercase{Mapeamento de ambientes com o robô Bellator}}
\documento{Análise tecnológica}

\autor{Luis Guilherme Machado Camargo}
\autordois{Pedro Alberto de Borba}
\autortres{Ricardo Farah}
\autorquatro{Stefan Campana Fuchs}
\autorcinco{Telmo Friesen}

\cita{CAMARGO, L.G.M; BORBA, P.A.; FARAH, R.; FUCHS, S.C.; FRIESEN, T}

\comentario{\UTFPRdocumentodata\ apresentada à Unidade Curricular de \UTFPRunidadedata\ do \UTFPRprogramadata\ da \ABNTinstituicaodata\ como requisito parcial para aprovação.}

\local{Curitiba}
\data{\the\year}

\begin{document}

\capa
\folhaderosto
\sumario
\chapter{Introdução}

O projeto apresentado neste documento trata-se do “Mapeamento de Ambientes com o robô Bellator” e é uma extensão do projeto “Bellator”. Ele teve sua última alteração em 2012 quando foi utilizado por Alexandre Jacques Marin, Júlio Cesar Nardelli Borges e Yuri Antin Wergrzn como plataforma de experimentos para o projeto final de conclusão de curso. O projeto para a disciplina de Oficina de Integração 3 será desenvolvido com base nesse robô. Na versão atual dele, está presente um conjunto de circuitos (com um microcontrolador) que gerencia as operações de baixo nível. Além disso, está presente um PC embarcado (executando o sistema Linux), que efetua as operações de alto nível.

A equipe deste projeto propõe modificar o robô Bellator para efetuar o mapeamento 2D de ambientes controlados como, por exemplo, labirintos construídos para fins de teste do robô. Posteriormente, em trabalhos futuros, ajustes finos poderão ser feitos para o uso em ambientes diversos, como escritórios, salas e quartos.

Na versão atual do Bellator, estão sendo utilizadas duas placas de circuito impresso – uma integrada com o microcontrolador e uma para a interface com os sensores – ambas ligadas por cabos entre si. Ao invés de produzir uma terceira placa para sensores adicionais (aspecto explicado mais à frente), o que aumentaria a quantidade de cabos, propõe-se desenvolver uma nova placa que realize a função de interface com todos os sensores e que seja acoplada ao microcontrolador. Este microcontrolador pode ser usado diretamente na forma encapsulada de circuito integrado (soldado diretamente na nova placa), ou integrado a um kit de desenvolvimento (acoplado como \textit{shield} na nova placa).

O sistema embarcado do robô será a placa de interface de sensores acoplada com o microcontrolador. Esse sistema realizará as funções de baixo nível, ou seja, leitura de sensores e controle do PWM dos motores. A estação base será um computador, provido de um software que efetua comunicação bidirecional com o robô. A estação será capaz de enviar comandos de movimentação (especificados manualmente pelo teclado) a ele, além de receber imagens da câmera e leituras dos sensores. No software, a partir das leituras dos sensores, será produzido um mapa em 2D simplificado do ambiente, com os obstáculos que forem detectados à medida que o robô andar, além do caminho estimado percorrido por ele. Protocolos de comunicação serão utilizados entre: circuito de baixo nível e o PC embarcado (através de porta serial), e entre PC embarcado e estação base (através de conexão WI-FI). A conexão entre a estação base e o robô deve ter um alcance de até 20 metros, e para isso a tecnologia WI-FI mostra-se adequada.

Um aspecto importante a ser notado é a exatidão e confiabilidade das medições de velocidade. No robô atual tem-se dois encoders, um para cada roda – a partir dos quais pode ser medida a velocidade e distância percorrida. Há certas desvantagens em utilizar essa abordagem, que são principalmente as questões de exatidão. Por exemplo, caso alguma roda escorregue, gire em falso ou sofra trepidações, as medições podem ser comprometidas – gerando distorções no mapa 2D. Por isso, propôe-se instalar novos sensores na carcaça do robô (acelerômetro e giroscópio) para adicionar maior confiabilidade nas medições do sistema – tendo em vista que esses sensores mensurarão o movimento real do robô e não somente o giro das rodas. Dessa forma, pode-se ter maior garantia de exatidão nos mapas gerados, levando-se em conta que a velocidade e posição do robô poderão ser melhor determinados. Especialmente em trabalhos futuros, se o robô for utilizado em ambientes acidentados ou em condições não ideais de terreno, esses sensores podem ser de grande valia – uma vez que nesses ambientes há maior chance da as rodas escorregarem, girarem em falso ou trepidarem.

\chapter{Especificação de Objetivos/Metas}
OBJETIVOS:

\begin{itemize}
  \item Implementar um software para comunicação de uma estação base (computador) com o robô, de forma que ela possa enviar comandos de movimentação ao robô, além de receber imagens da câmera e leituras dos sensores. Os comandos de movimentação (mover para frente, para trás, girar para esquerda/direita, parar) serão especificados por um utilizador humano através do teclado da estação base. 
  \item O meio de comunicação entre a estação base e o robô deverá ter alcance máximo de 20 m (se não houverem paredes ou obstáculos entre a estação base e o robô). Para isso a tecnologia WI-FI mostra-se adequada e, portanto, ela será utilizada.
  \item Inserir uma \textit{webcam} USB no robô, de modo que imagens do ambiente possam ser transmitidas à estação base. O propósito das imagens será unicamente permitir a visualização (pelo usuário da estação base, em tempo real) do ambiente no qual o robô está localizado. A câmera será conectada na porta USB do computador embarcado, e a transmissão de imagens será feita pelo canal Wi-Fi entre a estação base e o robô (o mesmo canal utilizado para a trasmissão de dados dos sensores e comandos de movimentação).
  \item Implementar, no software utilizado na estação base, a geração de uma mapa em 2D com o caminho estimado percorrido pelo robô e os obstáculos detectados pelo mesmo. Os obstáculos serão representados a partir dos pontos em que houve detecção pelos sensores.
  \item Instalar novos sensores (acelerômetro e giroscópio) para efetuar as medições de velocidade e posicionamento do robô com maior exatidão do que pode ser feito atualmente com os \textit{encoders}. Ambos os sensores serão posicionados na carcaça do robô. Caso discrepâncias de medição entre os \textit{encoders}, acelerômetro e giroscópio sejam detectadas (por exemplo, em caso de escorregamento de rodas), atenuações de erros poderão ser feitas no \textit{software} da estação base.
%  A velocidade e deslocamento lineares instantâneos serão determinados a partir da integração numérica da aceleração linear (cujas amostras serão obtidas com o acelerômetro em intervalos de tempo discretos). A velocidade e deslocamento angular instantâneos serão determinados a partir da integração numérica da aceleração angular (cujas amostras serão obtidas com o giroscópio em intervalos de tempo discretos). A posição atual do robô será gradualmente atualizada na representação do mapa à medida em que as amostras de aceleração linear e angular forem recebidas na estação base.
  \item Desenvolver uma placa de circuito impresso que realize a função de interface com os sensores e que seja acoplada ao microcontroldador. Este microcontrolador pode ser usado diretamente na forma encapsulada de circuito integrado (sendo soldado diretamente na nova placa) ou integrado a um kit de desenvolvimento (acoplado como \textit{shield} na nova placa).
  \item Em caso de falha de comunicação entre o robô e a estação base, o robô deverá permanecer parado e aguardando a conexão ser reestabelecida.
\end{itemize}

METAS:
\begin{itemize}
  \item Concluir o trabalho com um prazo máximo de até 10 semanas. Incluindo planejamento, desenvolvimento, teste e documentação. 
  \item Não ultrapassar o orçamento inicial e o orçamento limite, detalhados posteriormente.
Desenvolver e manter um cronograma para que todos os integrantes da equipe tenham a possibilidade de trabalhar com o projeto sem causar prejuízos às outras matérias do curso.
\end{itemize}

\chapter{Premissas e restrições}
PREMISSAS:
\begin{itemize}
  \item Por ser utilizado o robô Bellator que já provém de trabalhos anteriores, infere-se que não haverá necessidade de haver gastos de tempo com consertos de equipamentos defeituosos ou correções de bugs no código fonte. Parte-se do pressuposto que o robô funciona de acordo com o que foi exposto nos relatórios anteriores.
  \item O robô é capaz de detectar obstáculos (paredes e objetos fixos de tamanho considerável que sejam maiores que ele) através dos sensores. A distância mínima para detecção é de 20cm e a máxima de 150cm.
  \item O robô é capaz de locomover-se em terrenos planos, não acidentados e em condições não severas.
  \item Pressupõe-se que o robô será disponibilizado para a equipe sem custos.
Podem ser utilizados os equipamentos e componentes diversos que já estejam disponíveis, com o objetivo de redução de custos.
\end{itemize}

RESTRIÇÕES:
\begin{itemize}
  \item O tempo disponível para a equipe é limitado, portanto muita atenção será dada às fases de planejamento e testes iniciais de modo a evitar imprevistos.
  \item A equipe deverá seguir um calendário previamente estabelecido, tendo o objetivo de evitar atrasos.
  \item O robô não será capaz de se locomover em terrenos acidentados, em escadas e similares.
  \item O robô não transportará cargas.
  \item O robô e a estação base não executarão algoritmos de roteamento ou mapeamento autônomo de ambientes. O controle de movimentação deverá ser feito obrigatoriamente por um usuário humano junto à estação base. O robô não fará nenhuma movimentação automática em caso de falha de conexão. Ele permanecerá parado aguardando a conexão ser reestabelecida.
  \item O robô e a estação base não serão capazes de efetuar mapeamento 3D.
  \item O robô e a estação base não irão armazenar automaticamente fotos ou vídeos dos ambientes explorados.
  \item O robô e o ponto de acesso WI-FI da estação base devem estar a uma distância máxima de 20 metros um do outro (supondo que não hajam paredes ou obstáculos). Caso contrário, não haverá garantias de que a comunicação entre a estação base e o robô seja funcional.
  \item Não serão usadas imagens do ambiente para a geração dos mapas.
  \item Os obstáculos não serão identificados quanto ao tipo ou forma. Serão apenas detectados pela sua presença.
\end{itemize}

\chapter{Designação do Gerente e da Equipe}
A equipe consiste de cinco integrantes. O gerente ocupou esta função com consentimento de todos.

GERENTE:
\begin{itemize}
  \item Luis Guilherme Machado Camargo.
\end{itemize}
COLABORADORES:	
\begin{itemize}
  \item Pedro Alberto de Borba, Ricardo Farah, Stefan Campana Fuchs, Telmo Friesen.
\end{itemize}


\chapter{Responsabilidades e Autoridade do Gerente}
\begin{itemize}
  \item O gerente poderá efetuar os gastos de valores estimados na análise de custos sempre informando os outros integrantes da equipe. Caso exista a necessidade de utilizar os valores previstos na margem de erro do orçamento, toda equipe deverá ser notificada e informada dos motivos.
  \item O gerente poderá liberar verba para um membro da equipe caso seja necessário. O gerente deverá registrar o valor gasto, o produto/serviço requerido e a pessoa que solicitou os recursos. Além disso, deve informar os outros membros da equipe sobre o fato.
  \item O gerente deverá atualizar o planejamento do projeto conforme exista a necessidade de mudanças, além de informar a equipe sobre o fato.
  \item O gerente deverá garantir que o projeto esteja progredindo conforme planejado.
  \item O gerente sempre deverá se portar educadamente a todos os membros da equipe.
  \item O gerente não tem poderes para efetuar a demissão de ninguém.
  \item O gerente tem o poder de tomar decisões em nome da equipe, preferencialmente considerando a opinião dos outros membros.
  \item O gerente tem o poder de intervir em qualquer conflito que ocorra internamente ou externamente à equipe.
  \item O gerente deve intermediar as reuniões da equipe.
  \item O gerente deve controlar as horas de trabalho da equipe e o cumprimento de prazos.
  \item O gerente deve falar em nome da equipe quando não for possível que toda ela o faça.
  \item O gerente deverá cobrar a escrita de documentação por todos os integrantes da equipe, de acordo com o que for desenvolvido por cada um.
\end{itemize}

\chapter{Planejamento de Riscos}

%Neste capítulo estão expostos os riscos previstos para o projeto.

\includepdf[pages=1, pagecommand=]{Planejamento_de_Riscos_v3.pdf}
\includepdf[pages=2-9, pagecommand=]{Planejamento_de_Riscos_v3.pdf}

\input{trabalhos_correlatos}
\chapter{Análise tecnológica}
Nesta seção está explicitada, primeiramente, uma visão geral do projeto. Em seguida, há uma discussão detalhada a respeito dos requisitos de cada parte fundamental (estação base, sistema de comunicação, sistema embarcado). Por fim há uma enumeração das alternativas tecnológicas pesquisadas e das escolhidas para o preenchimento dos requisitos.

\section{Visão geral do projeto}

O projeto, como foi idealizado, consiste em um robô controlado manualmente capaz de efetuar mapeamento 2D de ambientes. A estação base é um computador controlado e monitorado por um usuário humano. O utilizador será capaz de enviar comandos de movimentação ao robô (via teclado) e receber \textit{feedback} do seu posicionamento e dos obstáculos detectados por ele. Além disso, as imagens em tempo real de uma câmera posicionada no robô -- aspecto explicado mais à frente -- poderão ser visualizadas pelo utilizador.

O sistema de comunicação deverá ter alcance máximo de 20 metros. Visto que toda a comunicação entre a estação base e o sistema embarcado será feita por um único canal, a velocidade de transmissão de dados deve ser suficiente para o envio de comandos de movimentação ao robô, recebimento de dados de leituras de sensores e recebimento de imagens da câmera em tempo real.

O sistema embarcado é, primariamente, o robô. Ele deve ser capaz de se mover para frente e para trás e girar para a esquerda e direita -- em velocidades não muito altas, o que é suficiente, visto que velocidades elevadas dificultam o controle de movimentação pelo usuário. Uma visualização em tempo real do ambiente pelo usuário, tendo o objetivo de facilitar o controle de movimentação manual, poderá ser feita através de imagens geradas por uma câmera fixa instalada no robô. 

O robô deve ser capaz de obter dados para cálculos (na estação base) da sua velocidade e deslocamento. Erros de medição em decorrência de escorregamento ou trepidação de rodas devem ser atenuados, visando dessa forma a futura utilização do robô em condições não ideais de terreno. Obstáculos próximos -- em uma distância mínima de 30 cm e máxima de 150 cm -- devem ser detectados de modo a possibilitar a confecção do mapa 2D em tempo real na estação base.

\section{Requisitos}


\subsection{Estação base}
%
%TODO verificar objetivos do projeto
Esta seção descreve os requisitos da estação base, que foram elaborados de forma a satisfazer os objetivos do projeto.

\begin{itemize} %-------------------

  \item O \textit{software} será executado em um computador pessoal.
    \begin{itemize}
      \item O programa deverá ser executado razoavelmente em computador com recursos de \textit{hardware} comparável aos padrões atuais (pelo menos 2GB de memória RAM DDR2 ou melhor e processador \textit{dual-core}).
      \item O \textit{software} deverá ser multiplataforma, ou seja, executar em diferentes sistemas operacionais (no mínimo Linux e Windows).
      \item Preferencialmente bibliotecas e ferramentas livres (e gratuitas) deverão ser utilizadas no desenvolvimento do \textit{software}.
    \end{itemize}

  \item O \textit{software} deve possuir uma interface gráfica.
    \begin{itemize}
      \item Um utilizador, através da interface gráfica, será capaz de controlar o robô enviando comandos de movimentação especificados pelo teclado. 
      \item O usuário receberá a imagem em tempo real (preferencialmente com atrasos não muito consideráveis) de uma câmera fixa instalada no robô. 
      \item Os dados instantâneos de velocidade e posição do robô serão mostrados ao usuário na interface gráfica.
      \item Um mapa 2D do caminho percorrido e dos obstáculos detectados pelo robô será gerado, na interface gráfica, à medida em que houver movimentação do mesmo. O caminho percorrido pelo robô será representado visualmente por pontos, gradualmente posicionados no mapa. Os obstáculos serão representados por marcações nas localidades onde houve detecção de objetos por sensores. Todos os pontos representados no mapa serão gerados a partir de amostras obtidas em intervalos de tempo discretos.
      \item O mapa 2D gerado na interface poderá ser salvo em um arquivo, podendo ser posteriormente carregado.
    \end{itemize}

\end{itemize} %----------------------



\subsection{Sistema de comunicação}
Esta seção descreve os requisitos do sistema de comunicação entre a estação base e o sistema embarcado.

\begin{itemize} %----------------------

  \item Distância entre robô e estação base.
    \begin{itemize}
      \item O sistema de comunicação deve possuir, ao menos, alcance máximo de 20 metros sem fios -- de modo que ambientes de tamanho razoável possam ser mapeados.
    \end{itemize}

  \item Velocidade e direção do fluxo de transmissão de dados.
    \begin{itemize}
      \item Toda a comunicação entre a estação base e o sistema embarcado será feita por um único canal e, portanto:
      \item A velocidade de transmissão do canal de comunicação deve ser suficiente para o envio de comandos de movimentação ao robô, recebimento de dados de leituras de sensores e recebimento de imagens da câmera em tempo real; 
      \item O fluxo de dados deve ser bidirecional (\textit{full-duplex}).
    \end{itemize}

  \item Complexidade de implementação.
    \begin{itemize}
      \item A tecnologia utilizada para a comunicação deve permitir fácil utilização dos protocolos de transporte TCP e UDP, simplificando o desenvolvimento do protocolo da camada de aplicação.
    \end{itemize}

\end{itemize} %----------------------



\subsection{Sistema embarcado}
Esta seção descreve os requisitos do sistema embarcado (robô).

\begin{itemize} %----------------------

  \item Movimentação do robô.
    \begin{itemize}
      \item O robô deve ser capaz de mover-se para frente, para trás e girar para a esquerda e direita.
%      \item A velocidade de movimentação pode ser relativamente pequena. %%TODO verificar velocidade do bellator nas monografias anteriores
    \end{itemize}

  \item Controle de posicionamento e velocidade.
    \begin{itemize}
      \item O robô deve ser capaz de obter dados que permitam calcular sua velocidade (linear e angular) e posição atual (deslocamento e rotação em relação à posição inicial). Deve ser capaz de enviar os dados à estação base.
    \end{itemize}

  \item Detecção de obstáculos.
    \begin{itemize}
      \item O robô deverá ser capaz de detectar obstáculos próximos -- com distância de no mínimo 30 cm e no máximo 150 cm -- localizados ao seu redor, determinando a distância de cada objeto detectado.
    \end{itemize}

\end{itemize} %----------------------


\chapter{Análise de opções tecnológicas}
%Alternativas de hardware
\input{opcoes_tecnologicas-hw} 
%Alternativas de software
\input{opcoes_tecnologicas-sw}
\chapter{Plano do projeto}
\section{Cronograma}
O cronograma do projeto está presente em anexo, em um arquivo do OpenProj denominado ``Bellator.pod''.

\section{Deliverables}

Na Tabela \ref{tab:deliverables} estão expostos os \textit{deliverables} previstos ao longo do projeto.

\begin{table}[!h]
  \centering
  \caption{Relação dos entregáveis com seus respectivos responsáveis e prazos}
  \begin{tabular}{p{3cm}|p{4cm}||p{7cm}}
    \toprule
    \textbf{Dia}   & \textbf{Auxiliar de Gerenciamento} & \textbf{Deliverables} \\
    \hline
    13/03/2013 & Stefan Campana Fuchs & 
    \begin{enumerate}[topsep=0pt, partopsep=0pt, itemsep=0pt]
      \item Modelo UML inicial do \textit{software}.
      \item Especificação inicial do protocolo de comunicação.
      \item Especificação inicial do diagrama esquemático do \textit{hardware}.
    \end{enumerate}\\
    \hline
    27/03/2013 & Telmo Friesen & 
    \begin{enumerate}[topsep=0pt, partopsep=0pt, itemsep=0pt]
      \item Demonstração de uma simulação de desenho de mapa, utilizando o \textit{software} da estação base.
      \item Demonstração do código do \textit{firmware} do 8051 portado para o ARM.
    \end{enumerate}\\
    \hline
    10/04/2013 & Ricardo Farah & 
    \begin{enumerate}[topsep=0pt, partopsep=0pt, itemsep=0pt]
      \item Demonstração de \textit{display} de imagens (na interface gráfica da estação base) de uma \textit{webcam} conectada ao computador.
      \item Placa impressa ou documentos que provem que a placa já está sendo impressa.
      \item Demonstração de um protótipo inicial (em \textit{protoboard}) da parte de baixo nível do sistema embarcado.
    \end{enumerate}\\
    \hline
    24/04/2013 & Stefan Campana Fuchs & 
    \begin{enumerate}[topsep=0pt, partopsep=0pt, itemsep=0pt]
      \item Demonstração da interface gráfica completa da estação base.
      \item Demonstração da comunicação entre estação base e sistema embarcado.
      \item Placa de baixo nível do sistema embarcado montada.
    \end{enumerate}\\
    \bottomrule
    \end{tabular}%
  \label{tab:deliverables}%
\end{table}%

\chapter{Or\c{c}amento Detalhado}

\begin{table}[!h]\tiny
  \centering
  \caption{Pre\c{c}o individual e total dos componentes do projeto}
    \begin{tabular}{rrccrr}
    \toprule
    \multicolumn{6}{c}{\textbf{Fornecedor: Digikey}} \\
    \multicolumn{1}{c}{\textbf{Item}} & \multicolumn{1}{c}{\textbf{Quantidade}} & \multicolumn{1}{c}{\textbf{Descri\c{c}\~ao}} & \multicolumn{1}{c}{\textbf{Descri\c{c}\~ao textual}} & \textbf{Preco unitário} & \textbf{Subtotal} \\
    \multicolumn{1}{c}{1} & \multicolumn{1}{c}{4} & \multicolumn{1}{c}{IC REG LDO 5V .95A SOT-223} & \multicolumn{1}{c}{REGULADOR 5V} & \$0,54 & \$2,16 \\
    \multicolumn{1}{c}{2} & \multicolumn{1}{c}{3} & \multicolumn{1}{c}{IC REG LDO 3.3V .95A SOT-223} & \multicolumn{1}{c}{REGULADOR 3V3} & \$0,54 & \$1,62 \\
    \multicolumn{1}{c}{3} & \multicolumn{1}{c}{4} & \multicolumn{1}{c}{IC REG LDO 1.8V .95A SOT-223} & \multicolumn{1}{c}{REGULADOR 1V8} & \$0,48 & \$1,92 \\
    \multicolumn{1}{c}{4} & \multicolumn{1}{c}{4} & \multicolumn{1}{c}{IC BUFF/DVR TRI-ST DUAL 20SOIC} & \multicolumn{1}{c}{BUFFER P/ PWM} & \$0,99 & \$3,96 \\
    \multicolumn{1}{c}{5} & \multicolumn{1}{c}{3} & \multicolumn{1}{c}{IC ARM7 MCU FLASH 32K 48-LQFP} & \multicolumn{1}{c}{LPC 2103} & \$6,16 & \$18,48 \\
    \multicolumn{1}{c}{6} & \multicolumn{1}{c}{3} & \multicolumn{1}{c}{IC BUFF/DVR SCHM TRG 6BIT 14SOIC} & \multicolumn{1}{c}{SCHMITT TRIGGER} & \$1,52 & \$4,56 \\
    \multicolumn{1}{c}{7} & \multicolumn{1}{c}{25} & \multicolumn{1}{c}{RES 47.0K OHM 1/8W 1\% 0805} & \multicolumn{1}{c}{RESISTOR PULL-UP 47K} & \$0,01 & \$0,23 \\
    \multicolumn{1}{c}{8} & \multicolumn{1}{c}{4} & \multicolumn{1}{c}{LED CHIPLED 645NM RED DIFF 0805} & \multicolumn{1}{c}{LED VERMELHO 1V8 20MA} & \$0,09 & \$0,36 \\
    \multicolumn{1}{c}{9} & \multicolumn{1}{c}{4} & \multicolumn{1}{c}{RES 20K OHM 1/8W 1\% 0805 SMD} & \multicolumn{1}{c}{RESISTOR LED 20K} & \$0,04 & \$0,16 \\
    \multicolumn{1}{c}{10} & \multicolumn{1}{c}{25} & \multicolumn{1}{c}{RES 22.0 OHM 1/8W 1\% 0805 SMD} & \multicolumn{1}{c}{22 OHMS P/ LIMITADOR DE VOLTAGEM} & \$0,01 & \$0,37 \\
    \multicolumn{1}{c}{11} & \multicolumn{1}{c}{12} & \multicolumn{1}{c}{DIODE ZENER DUAL 4.3V SOT-363} & \multicolumn{1}{c}{DUAL ZENER 4V3} & \$0,33 & \$4,01 \\
    \multicolumn{1}{c}{12} & \multicolumn{1}{c}{10} & \multicolumn{1}{c}{CAP CER 150PF 50V 1\% NP0 0805} & \multicolumn{1}{c}{CAPACITOR 150P FILTRO ENCODERS} & \$0,12 & \$1,24 \\
    \multicolumn{1}{c}{13} & \multicolumn{1}{c}{10} & \multicolumn{1}{c}{RES 332 OHM 1/8W 1\% 0805 SMD} & \multicolumn{1}{c}{RESISTOR 332 FILTRO ENCODERS} & \$0,02 & \$0,19 \\
    \multicolumn{1}{c}{14} & \multicolumn{1}{c}{10} & \multicolumn{1}{c}{CAP CER 33PF 50V 5\% NP0 0805} & \multicolumn{1}{c}{33P CAPACITOR P/ CRISTAL} & \$0,05 & \$0,53 \\
    \multicolumn{1}{c}{15} & \multicolumn{1}{c}{50} & \multicolumn{1}{c}{CAP CER 0.1UF 50V 5\% X7R 0805} & \multicolumn{1}{c}{0.1U PARA MAX3232 E REGULADORES} & \$0,06 & \$3,16 \\
    \multicolumn{1}{c}{16} & \multicolumn{1}{c}{12} & \multicolumn{1}{c}{CAP CER 10UF 10V 10\% X5R 0805} & \multicolumn{1}{c}{10U PARA REGULADOR} & \$0,13 & \$1,56 \\
    \multicolumn{1}{c}{17} & \multicolumn{1}{c}{4} & \multicolumn{1}{c}{IC DRVR/RCVR MULTCH RS232 16SOIC} & \multicolumn{1}{c}{MAX3232} & \$1,65 & \$6,60 \\
    \multicolumn{1}{c}{18} & \multicolumn{1}{c}{4} & \multicolumn{1}{c}{CRYSTAL 14.7456 MHZ 18PF SMD} & \multicolumn{1}{c}{14.7456MHZ 10PPM 18pF} & \$0,57 & \$2,28 \\
          &       &       &       &       & 53,388 \\
          &       &       &       & Frete: & \$37,67 \\
          &       &       &       & Total (\$): & 91,058 \\
          &       &       &       & IOF (\$): & \$5,81 \\
          &       &       &       & \textbf{Total (R\$):} & \textbf{R\$ 199,54} \\
          &       &       &       &       &  \\
    \multicolumn{6}{c}{\textbf{Fornecedor: Ebay}} \\
    \multicolumn{1}{c}{\textbf{Item}} & \multicolumn{1}{c}{\textbf{Quantidade}} & \multicolumn{1}{c}{\textbf{Descri\c{c}\~ao}} & \multicolumn{1}{c}{\textbf{Descri\c{c}\~ao textual}} & \textbf{Preco unitário} & \textbf{Subtotal} \\
    1     & 3     & MPU-6050 & Gyro e acelerometro de 3 eixos & \$8,78 & \$26,34 \\
          &       &       &       & Frete: & \$0,00 \\
          &       &       &       & Total (R\$): & R\$ 55,95 \\
          &       &       &       & IOF (R\$): & R\$ 3,56 \\
          &       &       &       & \textbf{Total (R\$):} & \textbf{R\$ 59,51} \\
          &       &       &       &       &  \\
    \multicolumn{6}{c}{\textbf{Fornecedor: 24 de maio}} \\
    \multicolumn{1}{c}{\textbf{Item}} & \multicolumn{1}{c}{\textbf{Quantidade}} & \multicolumn{1}{c}{\textbf{Descri\c{c}\~ao}} & \multicolumn{1}{c}{\textbf{Descri\c{c}\~ao textual}} & \textbf{Preco unitário} & \textbf{Subtotal} \\
    1     & 1     &       & Barra de pinos &       & R\$ 0,55 \\
    2     & 10    &       & Resistor &       & R\$ 0,50 \\
    3     & 10    &       & Resistor &       & R\$ 0,50 \\
    4     & 2     & Max 3232 & Line Driver/Receiver &       & R\$ 15,80 \\
    5     & 10    &       & Capacitor ceramico &       & R\$ 1,00 \\
    6     & 10    &       & Diodo zenner &       & R\$ 2,50 \\
    7     & 2     & 74hc244 & Buffer  &       & R\$ 2,60 \\
    8     & 1     & Genius 1.3mp Islim 1300 & Webcam  &       & R\$ 34,99 \\
    9     & 1     &	  & Placa  &       & R\$ 250,00 \\
          &       &       &       & \textbf{Total (R\$):} & \textbf{R\$ 308,44} \\
          &       &       &       &       &  \\
          &       &       &       & \textbf{Total geral(R\$):} & \textbf{R\$ 567,49} \\
    \bottomrule
    \end{tabular}%
  \label{tab:custos}%
\end{table}%




\raggedright
\bibliography{referencias}

\end{document}

