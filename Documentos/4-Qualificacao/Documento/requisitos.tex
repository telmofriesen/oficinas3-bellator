\section{Requisitos}


\subsection{Estação base}
\label{subsec:req_estacao-base}
%
%TODO verificar objetivos do projeto
Esta seção descreve os requisitos da estação base, que foram elaborados de forma a satisfazer os objetivos do projeto.

\begin{itemize} %-------------------

  \item O \textit{software} será executado em um computador pessoal.
    \begin{itemize}
      \item O programa poderá ser executado em computadores pessoais de desempenho médio (de acordo com os padrões atuais). Não haverá necessidade de uma máquina de alto desempenho e custo relativo para executar o \textit{software}. %Para referência, supôe-se que um computador com recursos de \textit{hardware} de desempenho médio equivale a pelo menos 2GB de memória RAM DDR2 ou melhor e processador \textit{dual-core}.
      \item O \textit{software} primariamente será executado em um único sistema operacional, sendo este Linux ou Windows. É desejável que o desenvolvimento (incluindo a escolha das ferramentas) seja feito de forma a simplificar o uso multiplataforma do \textit{software} futuramente. 
      \item Preferencialmente bibliotecas e ferramentas livres (e gratuitas) deverão ser utilizadas no desenvolvimento do \textit{software}.
    \end{itemize}

  \item O \textit{software} deve possuir uma interface gráfica.
    \begin{itemize}
      \item Um utilizador, através da interface gráfica, será capaz de controlar o robô enviando comandos de movimentação especificados pelo teclado. 
      \item O usuário receberá a imagem em tempo real de uma câmera fixa instalada no robô. 
      \item Os dados instantâneos de velocidade e posição do robô serão mostrados ao usuário na interface gráfica.
      \item Um mapa 2D do caminho percorrido e dos obstáculos detectados pelo robô será gerado, na interface gráfica, à medida em que houver movimentação do mesmo. O caminho percorrido pelo robô será representado visualmente por pontos, gradualmente posicionados no mapa. Os obstáculos serão representados por marcações nas localidades onde houve detecção de objetos por sensores do robô. Todos os pontos representados no mapa serão gerados a partir de amostras obtidas em intervalos de tempo discretos.
      \item O mapa 2D gerado na interface gráfica poderá ser salvo em um arquivo, podendo ser posteriormente carregado.
    \end{itemize}

\end{itemize} %----------------------



\subsection{Sistema de comunicação}
Esta seção descreve os requisitos do sistema de comunicação entre a estação base e o sistema embarcado.

\begin{itemize} %----------------------

  \item Distância entre robô e estação base.
    \begin{itemize}
      \item O sistema de comunicação deve possuir, ao menos, alcance de 20 metros sem fios -- de modo que ambientes de tamanho razoável possam ser mapeados.
    \end{itemize}

  \item Velocidade e direção do fluxo de transmissão de dados.
    \begin{itemize}
      \item Toda a comunicação entre a estação base e o sistema embarcado será feita por um único canal sem fios e, portanto:
      \item A velocidade de transmissão do canal de comunicação deve ser suficiente para, simultaneamente, o envio de comandos de movimentação ao robô, recebimento de dados de leituras de sensores e recebimento de imagens da câmera; 
      \item O fluxo de dados deve ser bidirecional (\textit{full-duplex}).
    \end{itemize}

  \item Protocolo de transporte.
    \begin{itemize}
      \item A tecnologia utilizada para a comunicação deve permitir fácil utilização do protocolo de transporte TCP. Como as leituras de sensores devem obrigatoriamente ser recebidas na estação base na mesma ordem em que forem enviadas pelo robô (e também os comandos de movimentação enviados pela estação base devem chegar ao robô em ordem), o uso desse protocolo de transporte simplificará muito a implementação do protocolo de aplicação ponto a ponto. O TCP possui ainda outro aspecto interessante: além de garantir a ordem de chegada, existem mecanismos de detecção de perdas de pacotes -- que efetuam o reenvio destes caso necessário.
    \end{itemize}

\end{itemize} %----------------------



\subsection{Sistema embarcado}
\label{subsec:req_sistema-embarcado}
Esta seção descreve os requisitos do sistema embarcado (robô).

\begin{itemize} %----------------------

  \item Imagens instantâneas do ambiente.
    \begin{itemize}
      \item O robô, através de uma câmera fixa, deverá ser capaz de enviar à estação base imagens instantâneas do ambiente onde ele se encontra.
%      \item A câmera deverá possuir ser capaz de produzir imagens em resolução VGA (640x480), RGB 24 bits, a 30 FPS.
    \end{itemize}

  \item Movimentação do robô.
    \begin{itemize}
      \item O robô deve ser capaz de mover-se para frente, para trás e girar para a esquerda e direita.
%      \item A velocidade de movimentação pode ser relativamente pequena. %%TODO verificar velocidade do bellator nas monografias anteriores
    \end{itemize}

  \item Controle de posicionamento e velocidade.
    \begin{itemize}
      \item O robô deve ser capaz de obter dados que permitam calcular sua velocidade e posição atual (deslocamento e sentido em relação à posição inicial). Deve ser capaz de enviar os dados à estação base.
    \end{itemize}

  \item Detecção de obstáculos.
    \begin{itemize}
      \item O robô deverá ser capaz de detectar obstáculos próximos -- com distância de no mínimo 30 cm e no máximo 150 cm -- localizados ao seu redor, determinando a distância de cada objeto detectado.
    \end{itemize}

\end{itemize} %----------------------

